We define the function $g : \Hz \to \Hx$ as the solution to 
\begin{equation}\label{g_def}
    g(\v{z}) = J_{\alpha \v{F}}\left(-\v{P}^T\v{M}^T\v{z} + \v{P}^T\v{L} \v{P}g(\v{z})\right)
\end{equation}
and note that since $\v{P}^T\v{L}\v{P}$ is strictly block lower triangular, $g$ is well-defined, and can be found by iteratively finding the prox on $\partial f_i$, working in parallel when any set of prox functions have all of the input required by $\v{L}$.
\begin{theorem}
The sequence of iterates $(\v{z}^k)$ of Algorithm \ref{coupled_alg}, applied to problem \eqref{zero_coupled} over a set of $n$ convex, proper, lower semi-continuous functions $f_i$ with the coupling constraint defined by \eqref{coupling_set_def}, converges weakly to a point $\v{z}^*$ such that $g(\v{z}^*)$ is a solution for \eqref{zero_coupled}, and $g(\v{z}^k)$ weakly converges to $g(\v{z}^*)$.
\end{theorem}

\begin{proof}
Our proof proceeds by first establishing the correspondence between the null space of $\v{M}\v{P}$, $\v{W}\v{P}$, and $\v{Z}\v{P}$ and $\mathcal{N}$. 
We also establish that for $\v{x} \in \mathcal{N}$, $\v{P}^T(\v{L}-\I)\v{P}\v{x} \in \mathcal{N}^\perp$.
We then show that the operator $T_\v{F}(\v{z}) = \v{z} + \gamma \v{M}\v{P}g(\v{z})$ is $\gamma$-averaged non-expansive for (potentially $\mu$-strong) maximal monotone operator $\v{F} = \left(\partial f_1(x_1), \dots, \partial f_n(x_n)\right)$.
We next show that the existence of a minimizer for \eqref{zero_coupled} is equivalent to the existence of a fixed point of $T_{\v{F}}$, and we therefore have weak convergence of $(\v{z}^k)$ to some $\v{z}^* \in \text{Fix}\left(T_\v{F}\right)$ by the $\gamma$-average non-expansivity of $T_\v{F}$.
Finally, we show that $g(\v{z}^k)$ converges to a unique weak cluster point $\v{x}^* \in \mathcal{N}$ such that $\v{x}^*$ is a minimizer of \eqref{zero_coupled}.

%  correspondence between the null space of $\v{M}\v{P}$, $\v{W}\v{P}$, and $\v{Z}\v{P}$ and $\mathcal{N}$
We begin with the null space of $\v{M}\v{P}$, $\v{W}\v{P}$, and $\v{Z}\v{P}$.
For any $\v{x} \in \mathcal{N}$, $\v{P}\v{x} = \left(\1 \otimes y_1, \dots, \1 \otimes y_m\right)^T$ for some set of $y_k \in \HKk$. 
This follows from the equality contraints in $\mathcal{N}$ and the permutation structure of $\v{P}$.
By assumptions \eqref{oars} on $M_k$, $W_k$, and $Z_k$, 
\[
\nullspace{M_k} = \nullspace{W_k} = \nullspace{Z_k} = \1
\]
Therefore for $\v{x} \in \mathcal{N}$, 
\begin{align}
\v{W}\v{P}\v{x} &= \diag{\left(W_1, \dots, W_m\right)} \begin{pmatrix}
    \1 \otimes y_1 \\
    \vdots \\
    \1 \otimes y_m 
\end{pmatrix} &= 0 \\
% \v{Z}\v{P}\v{x} = 0 \\
\v{Z}\v{P}\v{x} &= \diag{\left(Z_1, \dots, Z_m\right)} \begin{pmatrix}
    \1 \otimes y_1 \\
    \vdots \\
    \1 \otimes y_m 
\end{pmatrix} &= 0 \\
% \v{M}\v{P}\v{x} &= 0
\v{M}\v{P}\v{x} &= \diag{\left(M_1, \dots, M_m\right)} \begin{pmatrix}
    \1 \otimes y_1 \\
    \vdots \\
    \1 \otimes y_m 
\end{pmatrix} &= 0 
\end{align}
Therefore 
\begin{align}
    \mathcal{N} &\subseteq \nullspace{M} \\
    \mathcal{N} &\subseteq \nullspace{W} \\
    \mathcal{N} &\subseteq \nullspace{Z} 
\end{align}
We also know that $W_k = M_k^T M_k$, so $W_k \succeq 0$. 
Therefore, since $\nullspace{W_k} = \text{span}(\1)$ and $Z \succeq W$,
\begin{align}
\nullspace{M_k} &= \text{span}(\1) \\
\lambda_2(W_k) &> 0 \\
\lambda_2(Z_k) &> 0
\end{align}
and for any $\v{y}_k \in \HKk^{r_k}$ such that $\v{y}_k \neq \1 \otimes y_k$ for some $y_k \in \HKk$, $\v{y}_k$ is not in the nullspace of $M_k$, $W_k$, or $Z_k$.

For any $\v{x} \notin \mathcal{N}$, at least one set of coupling constraints is not satisfied.
Let $\v{y} = \v{P}\v{x}$, and $\v{y}_k \in \HKk^{r_k}$ be a subelement of $\v{y}$ which does not satisfied the coupling constraint, and therefore is not in the span of the ones vector.
Then 
\begin{align}
\v{M}_k \v{y}_k &\neq 0 \\
\v{W}_k \v{y}_k &\neq 0 \\
\v{Z}_k \v{y}_k &\neq 0 
\end{align}
Therefore 
\begin{equation}
\nullspace{MP} = \nullspace{WP} = \nullspace{ZP} = \mathcal{N} 
\end{equation}
This also means that 
\begin{equation}\label{range_PM}
    \ran(P^T M^T) = \mathcal{N}^\perp.
\end{equation}

% We also establish that for $\v{x} \in \mathcal{N}$, $\v{P}^T(\v{L}-\I)\v{P}\v{x} \in \mathcal{N}^T$.
Next, we show that for $\v{x} \in \mathcal{N}$, $\v{P}^T(\v{L}-\I)\v{P}\v{x} \in \mathcal{N}^\perp$.
For all $\v{x}^\perp \in \mathcal{N}^\perp$ and $\v{x} \in mathcal{N}$, we have (since $\v{P}^T \v{P} = \I$)
\begin{align}
    \left\langle \v{x}^\perp, \v{x}, \right\rangle &= 0 \\
    \left\langle \v{P}\v{x}^\perp, \v{P}\v{x}, \right\rangle &= 0 
\end{align}
We also know that for $\v{y} = \v{P}\v{x}$, we have $\v{y} = \left(\1 \otimes y_1, \dots, \1 \otimes y_m\right)^T$ for some values $y_k \in \HKk$.
We also know that if $\left\langle \v{y}, \v{y}'\right\rangle = 0$, then $\v{x}' = -\v{P}^T \v{y}'$ is in $\mathcal{N}^\perp$.
By assumption, each $Z_k$ is symmetric and $Z_k\1=0$, so $\1^T Z_k \1 = 0$.
We also know by assumption that $Z_k = 2\I - L_k - L_k^T$, so $\1^T (L_k - \I)\1 = -\frac{1}{2}\1^T Z_k \1 = 0$.
This means that for each $k$, $\left\langle \1 \otimes y^1_k, (\v{L} - \I)(\1 \otimes y^2_k)\right\rangle = 0$.
For any $\v{x}^1 \in \mathcal{N}$ and $\v{y}^1 = \v{P}\v{x}^1 = \left(\1 \otimes y_1^1, \dots, \1 \otimes y_m^1\right)^T$ we then have 
\begin{align}
\v{y}' &= (\v{L} - \I)\v{P}\v{x}^1 \\ 
&= (\v{L} - \I)\v{y}^1 \\ 
&= \diag\left(L_1 - \I, \dots, L_m - \I\right) \begin{pmatrix}
    \1 \otimes y_1^1 \\
    \vdots \\
    \1 \otimes y_m^2
\end{pmatrix} \\ 
&= \left((L_1 - \I)(\1 \otimes y_1^1), \dots, (L_m -\I)(\1 \otimes y_m^1)\right)^T
\end{align}
For any $\v{x}^2 \in \mathcal{N}$ and $\v{y}^2 = \v{P}\v{x}^2 = \left(\1 \otimes y_1^2, \dots, \1 \otimes y_m^2\right)^T$, we then have
\begin{align}    
    \left\langle \v{x}^2, \v{P}^T (\v{L} - \I)\v{P}\v{x}^1 \right\rangle &= \left\langle \v{x}^2, \v{P}^T \v{y}' \right\rangle \\
    &= \left\langle \v{P}\v{x}^2, \v{y}' \right\rangle \\
    &= \left\langle \v{y}^2, \v{y}' \right\rangle \\
    &= \sum_{i=1}^m \left\langle \1 \otimes y_k^2, (L_k - \I)(\1 \otimes y_k^1) \right\rangle \\
    &= 0
\end{align}
Therefore 
\begin{equation}\label{LIx_in_NT}
\v{P}^T(\v{L}-\I)\v{P}\v{x} \in \mathcal{N}^\perp \quad \forall \v{x} \in \mathcal{N}
\end{equation}

% minimizer of zero_coupled equiv. to fixed point of T_F
We now show that $T_F$ is $\gamma$-averaged non-expansive.
By the definition of the resolvent and the maximal monotonicity of $\v{F}$, we know for any $\v{z}^1, \v{z}^2 \in \Hz$, $\v{x}^1 = g(\v{z}^1)$, and $\v{x}^2 = g(\v{z}^2)$ that
\begin{align}
    \v{F}(\v{x}^1) &\ni -\v{P}^T \v{M}^T\v{z}^1 + \left(\v{P}^T \v{L} \v{P} - \I \right)\v{x}^1 \\
    \v{F}(\v{x}^2) &\ni -\v{P}^T \v{M}^T\v{z}^2 + \left(\v{P}^T \v{L} \v{P} - \I \right)\v{x}^2.
\end{align}
Let 
\begin{align}
\v{z} &= \v{z}^1 - \v{z}^2\\
\v{x} &= \v{x}^1 - \v{x}^2 \\
\v{z}^+ &= T_{\v{F}}\left(\v{z}^1\right) - T_{\v{F}}\left(\v{z}^2\right).
\end{align}
Then for maximal monotone $\v{F}$ (with $\mu = 0$) and maximal $\mu$-strongly monotone $\v{F}$ (with $\mu > 0$), we have
\begin{align}
    \left\langle \v{x}^1 - \v{x}^2, -\v{P}^T \v{M}^T \v{z}^1 + \left(\v{P}^T \v{L} \v{P}  - \I\right) \v{x}^1 \right\rangle &\\
    - \left\langle \v{x}^1 - \v{x}^2, -\v{P}^T \v{M}^T \v{z}^2 + \left(\v{P}^T\v{L}\v{P} - \I\right) \v{x}^2 \right\rangle &\geq \mu \norm{\v{x}^1 - \v{x}^2}^2\\
    \implies \left\langle \v{x}, -\v{P}^T\v{M}^T \v{z} + \left(\v{P}^T\v{L}\v{P} - \I\right) \v{x} \right\rangle & \geq \mu \norm{\v{x}}^2.\label{first}
\end{align}
We note that $\I = \v{P}^T \I \v{P}$, so 
\begin{align}
    \left\langle \v{x}, \left(\v{P}^T\v{L}\v{P} - \I\right) \v{x} \right\rangle &= \left\langle \v{x}, \v{P}^T(\v{L}-\I)\v{P}\v{x} \right\rangle \\
    &= -\frac{1}{2}\left\langle \v{x}, \v{P}^T \v{Z} \v{P} \v{x}\right\rangle.
\end{align}
We define $\alpha_\v{x}$ as
\begin{equation}
    \alpha_\v{x} = \begin{cases}
        1 \quad &\text{if} \, \v{W}\v{P}\v{x} = 0 \\
        \frac{\left\langle \v{P}\v{x}, \v{Z}\v{P}\v{x} \right\rangle}{\left\langle \v{P}\v{x}, \v{W}\v{P}\v{x}\right\rangle} & \text{otherwise}
    \end{cases}.
\end{equation}
We note that since $Z \succeq W$, we always have $\alpha_\v{x} \geq 1$.
Therefore the left-hand side of \eqref{first} reduces to
\begin{align}
& \left\langle \v{x}, -\v{P}^T\v{M}^T \v{z}\right\rangle - \frac{1}{2}\left\langle \v{x}, \v{P}^T \v{Z} \v{P} \v{x}\right\rangle \\
=& \left\langle \v{x}, -\v{P}^T\v{M}^T \v{z}\right\rangle - \frac{\alpha_\v{x}}{2}\left\langle \v{x}, \v{P}^T \v{W} \v{P} \v{x}\right\rangle.
\end{align}
We let $\norm{\v{W}}$ be the operator norm on $\v{W}$, and note that $\norm{\v{W}} = \norm{W}$.
We also know that $\norm{W} = \max_{k} \norm{W_k}$ since $W$ is block diagonal.
Let $\beta = \norm{W} = \max_{k} \norm{W_k}$.
Then, on the right-hand side, we have
\begin{align}
    \norm{\v{x}}^2 &\geq \frac{1}{\beta}\left\langle\v{x}, \v{P}^T\v{W}\v{P}\v{x}\right\rangle\\
    &\geq \frac{1}{\beta}\left\langle\v{x}, \v{P}^T\v{W}\v{P}\v{x}\right\rangle.
\end{align}
We therefore have
\begin{align}
    \left\langle \v{x}, -\v{P}^T\v{M}^T \v{z}\right\rangle - \frac{\alpha_\v{x}}{2}\left\langle \v{x}, \v{P}^T \v{W} \v{P} \v{x}\right\rangle &\geq \frac{\mu}{\beta}\left\langle\v{x}, \v{P}^T\v{W}\v{P}\v{x}\right\rangle \\
    \left\langle \v{x}, -\v{P}^T\v{M}^T \v{z}\right\rangle - \left(\frac{\alpha_\v{x}}{2} + \frac{\mu}{\beta}\right)\left\langle \v{x}, \v{P}^T \v{W} \v{P} \v{x}\right\rangle &\geq 0\\
    \left\langle -\v{M}\v{P}\v{x}, \v{z}\right\rangle - \left(\frac{\alpha_\v{x}}{2} + \frac{\mu}{\beta}\right)\left\langle \v{M}\v{P}\v{x}, \v{M}\v{P} \v{x}\right\rangle &\geq 0.\label{second}
\end{align}
By the definition of $\v{z}^+$, $\v{M}\v{P}\v{x} = \frac{\v{z}^+ - \v{z}}{\gamma}$, so we have
\begin{equation}
    \frac{1}{\gamma}\left\langle \v{z}^+ - \v{z}, \v{z}\right\rangle - \left(\frac{\alpha_\v{x}}{2} - \frac{\mu}{\beta}\right)\frac{1}{\gamma^2}\norm{\v{z}^+ - \v{z}}^2 \geq 0.
\end{equation}
By the results of \cite[Appendix A, Theorem 1, eq. 60-61]{bassett2024optimaldesignresolventsplitting}, applying the parallelogram law, we have 
\begin{equation}
    \norm{\v{z}}^2 - \frac{\gamma - \alpha_\v{x} - \frac{2\mu}{\beta}}{\gamma}\norm{\v{z} - \v{z}^+}^2 \geq \norm{\v{z}^+}^2
\end{equation}
and $T_\v{F}$ is $\gamma$-averaged non-expansive for $\gamma \in \left(0, \alpha + \frac{2\mu}{\beta}\right)$.
Because $\alpha_\v{x} \geq 1$, this condition is always satisfied for $\gamma \in (0, 1)$.

% Fixed point <=> argmin
We now show that the existence of a minimizer for \eqref{zero_coupled} implies the existence of a fixed point of $T_\v{F}$.
Suppose $\v{x}^* \in \argmin_{\v{x} \in \mathcal{N}} \sum_{i=1}^n f_i(x_i)$.
We then have some $w_i \in \partial f_i(x_i^*)$ given by $w_i = \left(w_{ik_1}, \dots, w_{ik_{p_i}}\right)$ for $k_1 \dots k_{p_i} \in K_i$ such that 
\begin{equation}\label{optimality_subgrad_sum}
    \sum_{i \in I_k} w_{ik} = 0 \quad k=1 \dots M
\end{equation}
Let $\v{w} = \left(w_1, \dots, w_n\right)$.
Condition \eqref{optimality_subgrad_sum} is equivalent to $\v{w} \in \mathcal{N}^\perp$.
By \eqref{range_PM}, we know that $\v{w} \in \ran(\v{P}^T\v{M}^T)$.
Since $\v{x}^* \in \mathcal{N}$, we also know that $\v{P}^T(\v{L} - \I)\v{P}\v{x} \in \ran(\v{P}^T\v{M}^T)$.
Therefore there exists a value $\v{z}^* \in \Hz$ such that 
\[-\v{P}^T \v{M}^T \v{z}^* = \v{w} + \v{P}^T(\I-\v{L})\v{P}\v{x}^*\].
By the definition of the resolvent, we then have
\[
\v{x}^* = J_\v{F}\left(-\v{P}^T \v{M}^T \v{z}^* + \v{P}^T \v{L}\v{P}\v{x}^*\right) = g(\v{z}^*).
\]
We further note that $\v{x}^* \in \nullspace{\v{M}\v{P}}$, so $\v{z}^* = T_\v{F}(\v{z}^*)$ and $\v{z}^* \in \text{Fix}(T_\v{F})$.

Since $T_\v{F}$ is $\gamma$-averaged non-expansive, for any starting point $\v{z}^0$ and sequence $(\v{z}^k)$ defined by $\v{z}^{k+1} = T_\v{F}(\v{z}^k)$, we know that $\v{z}^{k+1}-\v{z}^k \to 0$ and $(\v{z}^k)$ converges weakly to some $\v{z}^* \in \text{Fix}(T_\v{F})$ \cite[Prop 5.15]{bauschke_combettes}. 
This also implies the boundedness of $(\v{z}^k)$ by \cite[Prop 2.40]{bauschke_combettes}.

By the argument in \cite[Appendix 1, Theorem 1, Eq. 64-68]{bassett2024optimaldesignresolventsplitting} the non-expansivity of the resolvent on $\v{F}$ and the boundedness of $(\v{z}^k)$ imply the boundedness of $(g(\v{z}^k))$.
Therefore $(g(\v{z}^k))$ has a weak sequential cluster point $\v{\tilde{x}}$ \cite[Fact 2.27]{bauschke_combettes}.
Let $\v{x}^k = g(\v{z}^k)$ and $(\v{x}^{k_n})$ be a subsequence of $(\v{x}^k)$ converging weakly to $\v{\tilde{x}}$.
We next show that $\v{\tilde{x}}$ is unique by showing that $\v{\tilde{x}} = J_\v{F}\left(-\v{P}^T \v{M}^T \v{z}^* + \v{P}^T \v{L}\v{P}\v{\tilde{x}}\right)$.
We begin by showing that $\v{\tilde{x}} \in \mathcal{N}$.
We know that $\v{z}^{k+1} - \v{z}^k \to 0$ and $\v{z}^{k+1} = \v{z}^k + \gamma \v{M} \v{P}\v{x}$, so $\lim_{k\to\infty} \v{M}\v{P}g(\v{z}^k) = 0$.
Therefore, $\v{M}\v{P}\v{\tilde{x}} = \lim_{n \to \infty} \v{M}\v{P}\v{x}^{k_n} = 0$.
Since $\v{\tilde{x}} \in \nullspace{\v{M}\v{P}}$, we have $\v{\tilde{x}} \in \mathcal{N}$.
We now show that $\v{\tilde{x}} = J_\v{F}\left(-\v{P}^T \v{M}^T \v{z}^* + \v{P}^T \v{L}\v{P}\v{\tilde{x}}\right)$ by showing that $\v{\tilde{y}} = \v{P}^T \v{M}^T \v{z}^* + \v{P}^T(\v{L} - \I)\v{P}\v{\tilde{x}} \in \v{F}(\v{\tilde{x}})$.
Given the maximal monotonicity of $\v{F}$, we can show that $\v{\tilde{y}} \in \v{F}(\v{\tilde{x}})$ if, for any $\v{u} \in \text{Dom}(\v{F})$ and $\v{v} \in \v{F}(\v{u})$, $\left\langle \v{\tilde{x}} - \v{u}, \v{\tilde{y}} - \v{v} \right\rangle \geq 0$.
By the monotonicity of $\v{F}$ and the definition of $\v{x}^k$, we know that
\begin{equation}
    \left\langle \v{x}^{k_n} - \v{u}, -\v{P}^T \v{M}^T \v{z}^{k_n} + \v{P}^T(\v{L} - \I)\v{P}\v{x}^{k_n} - \v{v}\right\rangle \geq 0.
\end{equation}
Taking the limit as $n \to \infty$ we have 
\begin{align}
    \lim_{n \to \infty} &\left\langle -\v{M}\v{P}\v{x}^{k_n}, \v{z}^{k_n} \right\rangle - \left\langle \v{u}, -\v{P}^T \v{M}^T \v{z}^{k_n} \right\rangle \\
    &+ \left\langle \v{x}^{k_n} , \v{P}^T (\v{L} - \I) \v{P} \v{x}^{k_n} \right\rangle - \left\langle \v{u} , \v{P}^T (\v{L} - \I) \v{P} \v{x}^{k_n} \right\rangle \\
    & + \left\langle \v{x}^{k_n} , -\v{v} \right\rangle + \langle \v{u}, \v{v} \rangle
\end{align}
Looking at the limit terms individually, we have
\begin{align}
    \lim_{n \to \infty} \left\langle -\v{M}\v{P}\v{x}^{k_n}, \v{z}^{k_n} \right\rangle &= 0 \label{lim_1}\\
    \lim_{n \to \infty} \left\langle \v{u}, -\v{P}^T \v{M}^T \v{z}^{k_n} \right\rangle &= \left\langle \v{u}, -\v{P}^T \v{M}^T \v{z}^* \right\rangle \label{lim_2}\\
    \lim_{n \to \infty} \left\langle \v{x}^{k_n} , \v{P}^T (\v{L} - \I) \v{P} \v{x}^{k_n} \right\rangle &= \left\langle \v{\tilde{x}} , \v{P}^T (\v{L} - \I) \v{P} \v{\tilde{x}} \right\rangle \label{lim_3}\\
    \lim_{n \to \infty} \left\langle \v{u} , \v{P}^T (\v{L} - \I) \v{P} \v{x}^{k_n} \right\rangle &= \left\langle \v{u} , \v{P}^T (\v{L} - \I) \v{P} \v{\tilde{x}} \right\rangle \label{lim_4}\\
    \lim_{n \to \infty}  \left\langle \v{x}^{k_n} , -\v{v} \right\rangle &= \left\langle \v{\tilde{x}} , -\v{v} \right\rangle. \label{lim_5}
\end{align}
Limit \eqref{lim_1} follows from $\v{M}\v{P}\v{x}^k \to 0$ and the weak convergence of $(\v{z}^k)$.
Limit \eqref{lim_2} follows from the weak convergence of $(\v{z}^k)$.
Limit \eqref{lim_3} follows from the weak convergence of $(x^{k_n})$ and the fact that $\v{P}^T(\v{L} - \I)\v{P}$ is a bounded linear operator.
Limits \eqref{lim_4} and \eqref{lim_5} likewise follow from the weak convergence of $(\v{x}^{k_n})$.

We therefore have 
\[ \left\langle \v{\tilde{x}} - \v{u}, -\v{P}^T\v{M}^T\v{z}^* + \v{P}^T(\v{L} - \v{I})\v{P}\v{\tilde{x}} - \v{v} \right\rangle = \left\langle \v{\tilde{x}} - \v{u}, \v{\tilde{y} - \v{v}} \right\rangle \geq 0
\]
and $\v{\tilde{y}} \in \v{F}(\v{\tilde{x}})$.

We now have $\v{\tilde{x}} = J_{\v{F}}\left(-\v{P}^T\v{M}^T\v{z}^* + \v{P}^T\v{L}\v{P}\v{\tilde{x}}\right)$, and establish that $\v{\tilde{x}}$ is unique by the uniqueness of the resolvent on $\v{F}$.
Since $(\v{x}^k)$ has a unique weak cluster point $\v{\tilde{x}}$, it therefore weakly converges to $\v{\tilde{x}}$.
Furthermore, since $\v{\tilde{x}} \in \mathcal{N}$ and $\v{\tilde{y}} = \v{P}^T \v{M}^T \v{z}^* + \v{P}^T(\v{L} - \I)\v{P}\v{\tilde{x}} \in \mathcal{N}^\perp$ and $\v{\tilde{y}} \in \v{F}(\v{\tilde{x}})$, $\v{\tilde{x}} \in \argmin_{\v{x} \in \mathcal{N}} \sum_{i=1}^n f_i(x_i)$.
\end{proof}