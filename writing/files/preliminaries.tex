\subsection{Notation}
Notation
$\HH$
$\text{Dom}$
$\1$
$\ran$
$\succeq$
$\mathbb{S}^n$
$\Sp^n$
$\prox$
$\bigotimes_{i=1}^n \HH_n$
$\v{M}$
$\v{x}$
$\v{F}$
$J_{\v{F}}$ (and resolvent)
maximally monotonicity and maximal monotonicity of $\v{F}$
$\diag$
Attouch-Th\'era Dual ??
% In addition to the convergence of the $\v{x}$ iterates in \eqref{n_iteration}, the $\v{v}$ iterates can be used to compute the Attouch-Th\'era dual solution of a lifted version of the problem \eqref{zero_in_monotone}. In the following theorem, we denote by $\Delta$ the subspace
% $$\Delta = \{\v{x} \in \mathcal{H}^{n}|\;\v{x} = \1 \otimes x \text{ for some } x \in \mathcal{H}\}.$$
% \begin{theoremrep}\label{attouch-thera}
% Let $\v{v}^{*}$ and $\v{x}^{*}$ be limits of the algorithm \eqref{n_iteration}. Define $\v{u}^{*} = \v{v}^{*} + (\v{L} - \I) \v{x}^{*}$. Then $\v{u}^{*}$ is the solution to the Attouch-Th\'era dual for the problem
% \begin{equation} \label{at_primal}
%  \find_{\v{x} \in \HH^n} 0 \in \left(\v{A} + \partial \iota_\Delta\right)\v{x},
% \end{equation}
% which is,
% \begin{equation} \label{at_dual}
% \find_{\v{u} \in \HH^n} 0 \in \left(\v{A}^{-1} + \left(\partial \iota_\Delta\right)^{-\ovee} \right)\v{u}.
% \end{equation}
\subsection{Matrix-Parametrized Proximal Splitting}
The basis for our novel algorithm is the matrix-parametrized proximal splitting \cite{bassett2024optimaldesignresolventsplitting}, which can be given in reduced or full form as follows:

Assumptions on M, L, W, Z.
\begin{align}\label{oars}
    W &= M^T M \\
    Z &\succeq W \\
    \nullspace{W} &= \text{span}(\1)\\
    \text{diag}(Z) &=2\1 \\ 
    Z \1 &= 0 \\
    Z &= 2\I - L - L^T
\end{align} 
